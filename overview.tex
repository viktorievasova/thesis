\chapter{Overview}
\label{chap:overview} 

% Many years of researches resulted in several applications.
% The purpose of this section is to discuss currently available software packages that provide functionality similar to that of our application. 
% Since we do not know of any piece of software solving exactly 
In this chapter, we give an overview of software solutions providing functionality related to our area, namely the analysis of depth information and 3D reconstruction from photos and videos. 
% Since software packages solving the very same task are almost non-existent, we take a wider approach in Section 
Section \ref{soft} discusses software packages currently available to end-users, while Section \ref{lib} describes software libraries implementing relevant \cv\ algorithms. 
% we give an overview of available software dealing with the analysis of depth information and 3D reconstruction. 
% In the second part of this chapter, available programming languages and libraries considered for our work, are discussed.

\section{Available Software Packages}
\label{soft} 

% Software: 
%   - Canoma
%   - 123D 
%   - PhotoSynth 
%   - Matchmoving software (2d3 , Adobe 

\subsection{Matchmoving Software} 

% The task of reconstructing 3D information from photos or videos plays important role in several types of software packages. 
Matchmoving software in the film industry represents one of the earliest widely adopted commercial applications of algorithms that extract 3D information from 2D (video) imagery.
In such a software, accurate 3D information about the scene is only a secondary product and the user is mainly interested in obtaining information about the position and orientation the camera had at the time of capture of the individual video frames.
The knowledge of these parameters allows artists to add special effects and/or other synthetic elements to the video footage. 

Although matchmoving (also called camera tracking) can be achieved using many different techniques, the prevailing method detects easily definable elements -- such as corners -- in a frame of the video and tracks their movement on the subsequent frames. 
The camera parameters are then calculated from the 2D movement of the tracks. 
In \cv\, this approach is called \term{structure from motion}, since the structure of the scene (for example, the trajectory of the camera) is determined by the apparent movement of the tracks on the 2D frames.  
The 3D positions of the scene-points corresponding to the detected corners can also be calculated, giving the user a very rough point cloud reconstruction of the scene.

Examples of widely used matchmoving applications include 2d3's Boujou \todo{link} and Autodesk Matchmover \todo{link}.
The opensource libmv project \todo{link} aims to add matchmoving capabilities to the Blender 3D modelling application \todo{link}. 

\subsection{Microsoft PhotoSynth} 

Microsoft PhotoSynth, based on a research by Snavely, et al. \todo{citation}, has been publicly released in 2008. 
The software solution processes a set of unorganized pictures of a single scene and subsequently generates its 3D point cloud reconstruction. 
The main purpose of the reconstruction is to allow the user to navigate between the photos in a novel way, which respects the physical proximity of the cameras taking the individual photos. 
Perhaps most notably, the technology has been employed by the BBC and CNN. % TODO odkaz 
Recently, the possibility to generate $360^\circ$ panoramas and to upload the input photos from a Windows 8 mobile phone has been added. 

Microsoft PhotoSynth is a closed-source application with most of the computation running on Microsoft's servers. 
\todo{pridat jak photosynth postupuje, vcetne odkazu na Snavelyho clanky}

% One of the first applications that used to create a 3D model from a set of pictures of an object was Photosynth, designed by Microsoft in cooperation with the University of Washington \todo{citace jak na web PhotoSynthu tak na clanky Snavellyho; odstranil bych to, ze to byla jedna z prvnich aplikaci}. 
% The algorithm is based on pattern recognition and generates a 3D model of a scene photographed object including the point cloud. 
% After releasing the application, there were available only projects generated by Microsoft or BBC and later a cooperation with NASA was started. 
% Until two years later the version for public was released so users could upload own images to create a 3D model.

% In 2007, a one year after releasing Photosynth, Google introduced Street View to extend Google Maps and Google Earth. 
% At first, this additional application provided a panoramic views of cities in the USA, but soon it expanded to other places in the world.

\subsection{Autodesk 123D}

% Autodesk, an American corporation focused on 3D design software, released modelling application Autodesk 123D recently. 
% There are several additional tools available. 
Autodesk 123D is a bundle of several applications. 
One of them is 123D Catch, which creates a 3D model from a set of photos taken from different viewpoints.
The software is compatible with other Autodesk 123D applications, making it a viable solution for 3D artists who want to include real-world objects in their scenes.
The program is available for the Windows, Mac OSX, and iOS platforms.
To achieve good results, it is necessary to follow detailed instructions when taking the photos. 
% Otherwise, the resu
% The inclusion of blurred photos or untextured surfaces typically results in a failed attempt at reconstructing the scene. 
% For creating  3D model it is necessary to follow detailed instructions how to shot the pictures. 
% In most cases the process of building model fails because of wrong set of images. 
A failed reconstruction typically occurs when the photos are blurred, do not have solid background, or in the case of insufficient amount of photos. % TODO tim solid background si nejsem jisty

% TODO zvazit, kam/jestli dat nasledujici 
% If we evaluate accessibility of the software for mobile phones, Google Street View is running on every type of mobile platform without any larger errors. 
% There is a version of Photosynth for Windows phones and iOS operating system. 
% As we already mentioned, Autodesk developed a version for iOS as well. 
% But apparently, we miss applications developed for Android platform.

\section{Available Libraries}
\label{lib} 

We now briefly introduce the main \cv\ or Computer Graphics libraries that provide implementation of the algorithms necessary to build our software. % TODO tohle neni zrovna idealni veta...

% The problem considered in this work is getting an image information from a set of pictures and its processing afterwards.
% To be able to program a software dealing with a task from the area of computer vision, it is necessary to be familiar with a library that supports work with images. 
% That kind of functions offers OpenCV library. 

\subsection{OpenCV} 

OpenCV \todo{link} is a cross--platform library originally developed by Intel. 
It provides an implementation of several hundreds of \cv\ related algorithms, 
including, e.g., camera calibration routines, image segmentation algorithms, clustering algorithms, and linear algebra solvers.
The library was originally written in pure C. 
However, in the recent years the development shifted towards C++. 
The library provides interfaces for C, C++, Python and Java and suports the Windows, Linux, Mac OS, iOS and Android platforms. 

% TODO tady bychom to chteli vic provazat se zbytkem textu, napriklad uvest, ze implementuje algoritmy popisovane dale v praci 

\subsection{OpenGL} 

\todo{mozna}

% For our work is important that support for C, C++, Python and the Android platform is included in the library. 
% OpenCV4Android offers us great equipment for image processing. 

















