\chapter{Overview}
As we have already mentioned, with tasks of 3D Computer Vision have been dealing 
many researchers last years. In the Czech Republic there are several successful 
institutes specialised in image processing such as Center for Machine Perception 
or Institute of Information Theory and Automation with Department of Image 
Processing or Pattern Recognition Department for example where the latest issues are discussed. 
Their cooperation with governmental institutions and higher authorities indicates the importance of 
this area of computer science beside the popularity of creating 3D models in the general public. 

However, many years of researches resulted in several applications.
In this part we will give an overview of available software dealing with 
analysing depth information and 3D reconstruction. 
%Also we will focus on approaches used in that kind of computer vision tasks that have 
%been introduced. 
In the second part of this chapter, available programming languages and libraries
 considered for our work, are discussed.

\section{Existing Software}
One of the first applications that were used to create a 3D model from a set of pictures of an object
was Photosynth designed by Microsoft company in cooperation with University of Washington. 
The algorithm is based on pattern recognition and generates a 3D model of a 
photographed object including the point cloud. After releasing the application there were available 
only projects generated by Microsoft or BBC and later a cooperation with NASA 
was started. Until two years later the version for public was released so users 
could upload own images to create a 3D model.

In 2007, a year after releasing Photosynth, Google introduced Street View to 
extend Google Maps and Google Earth. At first, this additional application provided a 
panoramic views of cities in the USA, but soon it expanded to other places in 
the world.

Autodesk, an American corporation focused on 3D design software, released 
modelling application Autodesk 123D recently. There are several additional tools 
available. One of them is 123D Catch that creates 3D model from a set of 
pictures taken from different view angles. This software is compatible with 
Autodesk 123D application, so it is advisable idea for designers who want to 
work with real–world objects in the virtual scene.
The program is available for these 
operating systems: Windows XP, Windows Vista, Windows 7, Mac OSX and iOS\@. It seems 
that for creating such a 3D model it is necessary to follow detailed 
instructions how to shot the pictures. In most cases the process of building model fails because of 
wrong set of images. An error can occur when pictures are blurred, the background 
is not solid or the amount of photos is not sufficient.

If we evaluate accessibility of the software for mobile phones, Google Street 
View is running on every type of mobile platform without any larger errors. There is a 
version of Photosynth for Windows phones and iOS operating system. As we already 
mentioned, Autodesk developed a version for iOS as well. But apparently, we miss 
applications developed for Android platform.


\section{Existing Libraries}
In this work, the main assignment is getting an image information from a set of 
pictures and its processing afterwards.
To be able to program a software dealing with a task from the area of computer 
vision, it is necessary to be familiar with a library that supports work with 
images. That kind of functions offers OpenCV library. 

OpenCV is a cross–platform library developed by Intel. It provides large scale 
of functions supporting image processing; classes for segmentation and 
recognition, blob detection and other 2D and 3D feature toolkits are available.

For our work is important that support for C, C++, Python and the Android 
platform is included in the library. OpenCV4Android offers us great 
equipment for image processing. 





















