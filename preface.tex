\chapter*{Introduction}
In the recent years, many researchers have been interested by the task of Computer Vision. 
Particularly the problem of 3D reconstruction is being investigated a lot. 
At this moment there is a great number of algorithms to solve problems in this area. 
Most of the approaches depend on the kind of input that is available (a set of pictures – determining is also how many pictures are taken, a video stream, etc.) and also on the output that we expect.

The work of many researches resulted in several online applications such as PhotoSynth by Microsoft or 
Google StreetView that is used by millions of people nowadays.

Meanwhile we could observe a large progress of telecommunication devices. 
In recent years, it became very common to own a smartphone. 
Especially mobile phones with Android platform are very popular. 
A built–in camera and large amount of applications is typical for such kind of telephone.

The goal of this work is to explore ways how to connect these two phenomena and to create an Android application that takes a set of pictures and visualizes the result of the reconstruction of the depth information. 
Due to the ambiguity of solving the task, we have to be aware of the fact that it is possible, that our application will be limited to only particular type of scenes. 

The first part of this work analyses the problem, describes available software and gives an overview of programming libraries and languages that were used. 
Secondly, we focus on the theoretical basics and introduced approaches connected to this topic.
The next section is denoted to the implementation of our application and finally we evaluate and benchmark our work.
\addcontentsline{toc}{chapter}{Introduction}

