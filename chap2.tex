\chapter{Related Work}
In this section we mention some of the approaches that have been proposed earlier. 
We consider algorithms for interest points detection and description especially. 
It gives us an overview of already existing work.
\section{Interest Point Detection}
In 1988, Chris Harris and Mike Stephens \cite{harris1988} published their corner detector based on combining corner and edge detector. 
Although this approach is not scale–invariant, it is widely used nowadays. 

However, many computer vision tasks deal with the real world input data images where objects appear in different ways depending on the scale. 
Due to this fact, Lindberg came out with automatic scale selection for feature detection \cite{lindberg1998}.

There are several other approaches that have been introduced such as edge–based region detector by Jurie and Schmid \cite{jurie2004} or salient region detector by Kadir and Brady \cite{kadir2001}.
But apparently they are not used that much. It seems that due better results and stability, most popular are Hessian–based detectors.

Another example of Hessian–based detector is a detector published by Herbert Bay et al.\  in Speeded–Up Robust Features \cite{surf2006}. 
This detector is partly inspired by SIFT algorithm, but it is several times faster and more robust against the scale transformation and the image rotation. 
It relies on the approximation of determinant of the Hessian matrix that can be computed using integral images. 
Hence, the computation is very fast (see the previous chapter).

\section{Interest Point Description}
Large number of interest point descriptors were introduced. 
The most common one is scale–invariant feature transform \cite{lowe1999}, SIFT for short. 
This algorithm was published in 1999 by a Canadian computer scientist David G. Lowe. 
The descriptor is computed from the intensities around the key point locations where the local gradient direction of picture intensities gives description of the local  key point.
