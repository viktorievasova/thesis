% notace pro zapis vektoru
\newcommand\vect[1]{#1}             

% zkratka pro boldovani 
\def\mathb{\mathbf}                 

% TODO pred odevzdanim prace je nutne zkontrolovat, ze toto makro se uz v textu nevyskytuje
\newcommand\todo[1]{(\textit{\underline{TODO:} #1.})}

% zkratka pro zvyrazneni nazvu tridy (pouzivane v kapitole OpenCV pro prehledne cteni o implementovanych tridach
\newcommand{\stype}[1]{\texttt{#1}}	

% TODO zkontrolujme, jestli to chceme psat velkymi nebo malymi pismeny 
\def\cv{Computer Vision} 
\def\pc{point-cloud}

% zvyrazneni nove definovaneho pojmu
\newcommand{\term}[1]{\textit{#1}}

% zakladni matematicke pojmy 
\def\R{\mathbf{R}} 
\def\Rn{\mathbf{R}^n} 
\def\Rm{\mathbf{R}^m} 
\def\x{\mathbf{x}}
\def\K{K}
\def\convolution{*}

% obskurnejsi matematicke zkratky 
\newcommand{\fpp}[1]{ \frac{ \partial^2 f }{ #1 } }
