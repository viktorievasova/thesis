% notace pro zapis vektoru
\newcommand\vect[1]{#1}             

% zkratka pro boldovani 
\def\mathb{\mathbf}                 

% TODO pred odevzdanim prace je nutne zkontrolovat, ze toto makro se uz v textu nevyskytuje
\newcommand\todo[1]{(\textit{\underline{TODO:} #1.})}
\newcommand\update[2]{(\textit{\underline{UPDATE:} #1.) #2}}

% zkratka pro zvyrazneni nazvu tridy (pouzivane v kapitole OpenCV pro prehledne cteni o implementovanych tridach
\newcommand{\stype}[1]{\texttt{#1}}	

% TODO zkontrolujme, jestli to chceme psat velkymi nebo malymi pismeny 
\def\cv{Computer Vision} 
\def\pc{point-cloud}

% zvyrazneni nove definovaneho pojmu
\newcommand{\term}[1]{\textit{#1}}

% zakladni matematicke pojmy 
\def\R{\mathbf{R}} 
\def\Rn{\mathbf{R}^n} 
\def\Rm{\mathbf{R}^m} 
\def\x{\mathbf{x}}
\def\xd{\mathbf{x'}}
\def\K{K}
\def\convolution{*}
\def\F{\mathbf{F}}
\def\T{\mathbf{T}}
\def\MI #1 {MI_{#1}}
\def\H #1 {H_{#1}}
\def\L{L(x, y; t)}
\def\func{f(x, y)}
\def\gauss{g(x, y; t)}

% obskurnejsi matematicke zkratky 
\newcommand{\fpp}[1]{ \frac{ \partial^2 f }{ #1 } }
\newcommand{\inftyint}{\int_{-\infty}^{+\infty}}	%integral
\newcommand{\imd}[1]{ \frac{ \partial I }{ #1 } } 	%image derivatives

% vypisovani kodu v textu
\definecolor{dkgreen}{rgb}{0,0.6,0}
\definecolor{gray}{rgb}{0.5,0.5,0.5}
\definecolor{mauve}{rgb}{0.58,0,0.82}
\definecolor{red}{rgb}{0.6,0,0}

\lstset{frame=tb,
  language=Java,
  aboveskip=6mm,
  belowskip=6mm,
  showstringspaces=false,
  columns=flexible,
  basicstyle={\small\ttfamily},
  numbers=none,
  numberstyle=\tiny\color{gray},
  keywordstyle=\color{mauve},
  commentstyle=\color{dkgreen},
  stringstyle=\color{red},
  breaklines=true,
  breakatwhitespace=true
  tabsize=3
}