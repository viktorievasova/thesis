
\chapter{Image Processing Algorithms}
\label{chap:algorithms}

In this chapter, we introduce the image processing algorithms often applied in software solutions with functionality similar to those discussed in Chapter \ref{chap:overview}. 
Applications reconstructing depth information from photographs or video sequences typically have at their core an algorithm that detects and tracks image primitives (such as corners, blobs, lines, individual pixels, etc.) across several photos/frames. 
This knowledge allows one to estimate the camera parameters, which in turn makes it possible to compute the spacial parameters of the tracked primitives.
% Tohle je asi moc velke rozpatlavani, takze jsem to ani nedopsal: For example, the solution outlined in \todo{reference na Snavelyho} extracts certain image features from individual input photos, matches these features between pairs of photos to obtain tracks, then proceeds to estimate camera parameters and 3D positions of points correponding to these features and finished by visualizing the resulting \pc.

The approaches to computing correspondences between pairs of photos can be divided into \term{sparse correspondence} and \term{dense correspondence} algorithms. 
In the former, a relatively conservative subset of only the most stable \term{features} are considered and paired. 
In the latter, the objective is to obtain a correspondence in the second image for (almost) every pixel of the first image. 
% Thus, dense correspondence algorithms allow us to obtain full diparity map of the first image in the image pair. 
% We can either aim to establish correspondences between a restricted set of primitives, for example just the most stable looking corners detected in the images, or we can aim to establish a correspondence from (almost) every pixel of the reference image. 
% The former approach is called \term{sparse correspondence problem} 

We describe two techniques solving the sparse correspondence problem -- the SIFT \cite{lowe1999} and SURF \cite{surf2006} feature detectors/descriptors -- and then discuss the Lucas-Kanade method for computing optical flow. % XXX ten optical flow je nuten do toho zasadit nejak vic 

% We describe three techniques -- the SIFT feature detector/descriptor, the SURF feature detector/descriptor, and the Graph-cut algorithm. \todo{potrebujeme urcite pridat citace}
% SIFT and SURF are the most used algorithms to detect features.
% Both of them are robust to scale and rotation.
Generally, sparse feature matching algorithms first extract features from each photo. % from a pair of photos of the same scene. % TODO nejsem si jisty, jestli tam napsat "of the same scene", protoze taky mohou byt pouzity k object detection 
If the feature detection is \term{repeatable}, most of the features should be detected in all images where the corresponding scene elements are visible.
Then, for each feature a \term{descriptor} is generated, typically a high-dimensional vector. 
This descriptor is constructed in a way ensuring invariance to various image transformations. 
For instance, the SIFT descriptor is invariant to translation, rotation, scaling, and image noise. 
Thus, applying any of those transformations to the image should leave the descriptors unaffected, making it possible to match image features using Euclidean metric. 
The SIFT descriptor is also partially invariant to illumination changes and to affine transformations, the latter enabling us to perform matching of images capturing the scene from slightly different viewpoints.
% Achieving full invariance to projective transformations is typically too computationally prohibitive.  
% Usually, a descriptor is affinely invariant, meaning .
% Hence for a feature from different viewpoints should be generated the same vector.
% Since the descriptors are computed, we can match them between a pair of images. 
% Due to the fact that our descriptors are vectors, we can match them according to the distance in the space, using Euclidean distance for example.
 
\section{Scale-Invariant Feature Transform}

The scale-invariant feature transform (SIFT) was introduced by Lowe in \cite{lowe1999}, building on a previous work by Lindeberg \cite{lindeberg1998}. 
The paper describes both a feature detector and descriptor. 
To this day, SIFT remains a popular choice for applications such as panorama stitching and object detection. 
It is able to detect objects in images even in the presence of clutter.

Let us outline the major steps in which the algorithm proceeds in its task to repeatably generate a set of features in an image and to compute their corresponding descriptors. 
Recall that an imporant and desirable property of these descriptors is their invariance to image transformations like rotation, scaling or illumination changes. 
% The algorithm starts by building the scale-space of the image. 
Firstly, candidate features are detected over all scales of the grayscale version of the input image. 
Locations of these features are then determined with subpixel precision. 
Simultaneously, features corresponding to unstable image regions are filtered out from the dataset. 
  % Then, for each potential feature, its location and scale model is determined with subpixel precision by fitting a quadratic function.
  % This quadratic functions is also used to filter out unstable features that lie on an edge or are likely to result from image noise. 
In the next stage, each feature is assigned its orientation and scale based on the prevailing gradient directions in the image neighbourhood. 
This orientation is used to establish a reference frame for the construction of the descriptor.
The last step is the feature description.
The SIFT feature descriptor is a 128-dimensional vector determined by the gradients around the feature at the corresponding scale. 
Since the computation of the descriptor is performed using the abovementioned local reference frame, 
the resulting descriptor is invariant to transformations that affect this reference frame in a covariant manner but only partially invariant to affine transformations and illumination changes.
We now proceed to describe both the detection and the descriptor construction steps in more detail.

The first step in the feature detection part of the algorithm is to identify candidate locations of all scales in a highly repeatable manner. 
For this, Gaussian scale-space is employed. 
Recall its definition from Section \ref{sec:gaussder}:
\[
L(x, y; \sigma) = g(x, y, \sigma) * f(x, y),
\]
where $\sigma$ is the deviation of the Gaussian function, $*$ the convolution operator, $f(x, y)$ is the input image, and $g(x, y)$ the Gaussian function
\[
 g(x, y, \sigma) = \left( \frac{1}{\sqrt{2\pi \sigma _1^{2}}} \cdot e^{-\frac{x^{2}+y^{2}}{2\sigma_1^{2}} } \right).
\]

To extract the features in the scale-space we detect the local extrema in the difference-of-Gaussian function $D(x, t, \sigma)$.
The difference-of-Gaussian (DoG) is defined as the difference of two neighbouring images of the scale-space: 
\[
 D(x, y, \sigma) = L(x, y; 2\sigma) - L(x, y; \sigma) = (g(x, y, 2\sigma) - g(x, y, \sigma)) * f(x, y).
\]
Each sample point of the DoG is compared to its eight neighbours in the current scale and nine neighbours on the levels below and above.
If it is larger than all of these neighbours, it is selected as a maximum; if it is smaller than all of them, it is selected as a minimum. 
It can be shown that the DoG approximates the Laplacian of the image at the particular location. 
These extrema should therefore correspond to ``blobs'' in the image.
The application of scale-space ensures that blobs of all possible sizes are detected. 
These locations are taken as an initial set of candidate feature. 

We need to filter this set significantly to ensure high repeatability of the resulting set of features. 
Usually, the detection of scale-space extrema results is a large number of candidate keypoints. 
However, some of them are local extrema resulting from image noise or the responses of the operator along edges. 
Such structures are unstable and should be discarded from the set of candidate keypoints.
To reject such unstable features, we fit a 3D quadratic function to the local sample points and interpolate the location of the extremum.
This is done using quadratic Taylor expansion of the function $D(x, y, \sigma)$ around the detected point.
By calculating the minimum we get the position of the local extremum with a subpixel precission.
Properties of this quadratic function are used to reject unstable extrema: if the function is too shallow, we discard the feature as a result of image noise; if it is too narrow, the extremum likely corresponds to an edge in the image.  
Otherwise we accept it as a feature.

% orientation:
To achieve invariance to rotation, local orientation must be determined for each feature. 
First, we compute a histogram of orientations of image gradients around the its location. 
The peaks of the histogram represent the dominant orientations. 
The orientation corresponding to the highest peak is assigned to the feature. 
If there are several peaks of magnitude within 80\% of the highest peak, multiple keypoints are generated -- one for each such orientation.

The descriptor is then constructed by considering several histograms in the four quadrants of the reference frame of the keypoint.
Typically, each quadrant is divided into four subregions. 
In each of the resulting sixteen subregions, we construct a histogram of orientations of gradients sampled regularly in the subregion. 
Concatenating the values in these histograms results in a 128-dimensional vector.

\section{Speeded-Up Robust Features}

Another popular feature detector/descriptor is the Speeded Up Robust Features (SURF) algorithm, introduced by Bay et al. \cite{surf2006}. 
It is influenced by the SIFT algorithm described above and is based on Hessian-matrix approximation and the computation of Haar wavelet responses. 
The major stages are to detect the features, estimate their orientation and then to each of them assign a descriptor.
Similarly to SIFT, keypoints are again detected in a scale-space to achieve invariance to scaling.
In this approach, however, the scale-space is not constructed explicitly and integral images are employed instead to decrease the computation time. 

The SURF feature detection is based on detecting blob structures using Gaussian kernels.
It uses the Hessian matrix of the convolution of the image with the second derivative of Gaussian.
The determinant of this Hessian expresses the local change around the area.
The points that are simultaneously local extrema of both the determinant and the trace of the Hessian matrix are chosen as candidate keypoints.
%The detection of interest points is performed using an approximation of the Hessian-matrix.

However, the convolution is very costly to calculation time.
It is approximated and speeded-up with the use of integral images and approximated kernels (\ref{sec:integral_images}).
In this way the algorithm is very fast.

The SURF algorithm does not build the scale-space by explicit down-sampling the images.
To detect the features across scales SURF uses larger and larger Gaussian kernels.
The larger Gaussian kernel is used, the more down-sampled image is investigated.

Similarly to SIFT, the SURF descriptor is a vector describing the distribution of intensity values within the neighbourhood of the keypoint. 
The difference lies mainly in the fact that the SURF descriptor is constructed using Haar wavelet responses around the point of interest. % TODO skutecne sum? mozna mi neni jasne, v jakem smyslu
The Haar wavelet responses can be calculated efficiently with integral images again.
Before the descriptor calculation, to be invariant to image rotation it is needed to determine the orientation. 
By determining a unique orientation for a interest point, it is possible to achieve rotational invariance. 
Before the descriptor is calculated the surrounding area of the detected feature is rotated to its direction.

The SURF descriptor describes the surrounding area of the feature.
This area is divided to 4 $\times$ 4 subareas.
To each of these subareas the Haar wavelet responses are calculated in the $x$ and $y$ directions and then are described by the vector
$$v = (\sum{d_x}, \sum{d_y}, \sum{|d_x|}, \sum{|d_y|}),$$
where  $d_x$ and $d_y$ are the wavelet responses in $x$ and $y$ directions.
Concatenating this for all $4 \times 4$ subregions gives $64$-dimensional vector descriptor.

Matching the SURF descriptors between two images can be speeded-up by including the trace of the Hessian (Laplacian) to the process of finding corresponding keypoints. 
We exploit the sign of Laplacian that distinguishes bright blobs on dark background from the reverse situation. 
In the matching stage we compare only features if they have the same sign – type of contrast. 
This costs no additional computation time since this quantity has already been computed at a previous stage.




\section{SIFT vs. SURF}
Both SIFT and SURF are popular feature descriptors based on the distribution of intensity around the interest point.
SURF has been shown to have similar performance to SIFT, but being much faster.
The work \cite{surf2006} highlights the fast computation of the descriptor.
The element enabling this speed-up are the approximations using integral images. 
Also, the SURF descriptor is a 64-dimensional vector, enabling faster computation of the Euclidean distance between the descriptors compared to the 128-dimensional SIFT descriptors.
However, the performance depends on many factors.
The results of comparing and testing SIFT against SURF can be found in \cite{siftsurf2001} and \cite{siftsurf2009}.

In this \cite{siftsurf2001} study the robustness of SIFT and SURF was tested under several different conditions.
It was shown that SIFT and SURF gives similar results when testing the robustness to rotation, illumination and noise 
(the performance of both SIFT and SURF rapidly  decreased on the images with ``salt and pepper" noise).
When testing the viewpoints, two images available in the dataset taken from different viewpoints were chosen and in this case SIFT outperformed SURF.
In general matching the results show that all SIFT and SURF descriptors perform well on all datasets.


In \cite{siftsurf2009} is shown that SIFT still outperform SURF in scale, rotation and blur, but is slow and not good at illumination changes.
SURF gives the best results in computational time and when testing illumination, but it is not stable to rotation and illumination changes. 
This shows that none of these detectors/descriptors is best for all deformation.
Choosing the method mainly depends on the application.

Both of the studies have similar results except the calculation time.
While in \cite{siftsurf2001} was SIFT the most efficient, in \cite{siftsurf2009} the SURF was the fastest.


% \section{Max-flow dense correspondence algorithm}

% In computer vision there is a general task of stereo correspondence.
% It is a problem of discovering the closest match between points of two pictures typically taken from different positions.
% Presumably, rectification of the images considerably simplifies the situation. 
% Sometimes this constraint can be done.
% Once the correspondences are solved, we obtain a disparity map and it can be exploited to reconstruct the positions and distances of the cameras.

% TODO: Nasledujici patri do uvodu kapitoly a taky jsem to tam presunul. 
% Generally, there are two approaches to find the corresponding points.
% One of them is to detect features and interest points in an image and then find corresponding points in the other one.
% Only these high distinctiveness points are matched, hence it is called sparse stereo matching.
% The second way, dense stereo matching, is to match as many pixels as possible.

% In 1999, Roy published an algorithm %\cite{roy1998} 
% \cite{roy1999} to solve the stereo correspondence problem formulated as a maximum-flow problem in a graph.
% It solves efficiently maximum-flow and gives a coherent minimum-cut that presents a disparity surface for the whole image.
% This way provides more accurate depth map then the classic algorithms and the result is computed in shorter time in addition.

%The matching space is defined as a projective 3-d volume (to allow pyramids as well as cubes) formed

\section{Lucas-Kanade algorithm for optical flow} 

Optical flow algorithms are typically applied in postprocessing software for movies to track an object through a video sequence. 
We can think of this problem as the image correspondence problem optimized for the situation where there is very little camera movement between the registered pair of photos, as is in the case with two consecutive frames of a video sequence. 
The output of such algorithm is an array of $2$-dimensional vectors $\mathbf{d}_\x$, where $\x$ is an image position, with the property that the point $\x$ on the first frame corresponds to the point $\x + \mathbf{d}_\x$ on the second one. 
Optical flow is typically not constrained by epipolar geometry, since the assumption is that the video is of a dynamic scene that includes objects moving relative to the camera and each other.
The algorithms for optical flow tend to work only when the differences between the images are significantly limited and the registration can be performed only on a small neighbourhood of the original position of a feature. 

We now describe a classical Lucas-Kanade method \cite{lucas} for calculating the optical flow between two images $I_1$ and $I_2$.

The method proceeds by specifying linear constraints on the values $\mathbf{d}_\x$ and then solves the resulting over-determined system in the least squares sense. 
For each pixel $\x$ that is being tracked, we consider the following equation: 
\begin{align} 
\label{optfloweq} 
\Delta I_1(\x) \cdot \mathbf{d}_\x = I_1(\x) - I_2(\x),
\end{align} 
where $\Delta I_1(\x)$ is the image gradient at $\x$.
% $$\Delta I_1(\x) = \Big(\imd1{\partial x}(\x), \imd1{\partial y}(\x)\Big).$$
This is motivated by the observation that the apparent movement should be roughly in the direction of the image gradient for locations with a large jump of the intensity value between the frames. 
On the other hand, when the intensity of a pixel is approximately the same in both frames, we either expect no movement or just a movement perpendicular to the image gradient.
However, the inclusion of just the above equations does not enforce spatial consistency of the movement in any way.
This is taken into account by repeating equation (\ref{optfloweq}) for each position $\x^i$ taken from the neighbourhood of $\x$ of a fixed size. 
Specifically, we add the following equation for each $\x^i$:
$$\Delta I(\x^i) \cdot \mathbf{d}_\x = I_1(\x^i) - I_2(\x^i).$$
Note that the differences are read from the position $\x^i$ while the movement constrained is of the pixel $\x$.
Solving the above over-determined system using conventional methods gives us the desired values $\mathbf{d}_\x$.
% Optical flow is calculated for every pixel of the video frame, meaning that the output is a field of $2$-dimensional vector $d_\x$ (for each pixel $\x$ of the image) determining the movement of the pixel between the two frames of the video. 

