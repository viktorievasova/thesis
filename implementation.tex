
\chapter{Implementation}
\label{chap:implementation}

This chapter is devoted to implementation of our software.
The result of this work is an Android application that allows to shoot pictures with a camera, browse them and pick photos for further processing.
The output is a 3D model representing the disparity map which was calculated on the base on the selected input data.
In the first section we enumerate the subtasks and chosen algorithms and then describe the structure of the software resources.
 
\section{Implementation outline}
To implement the Android application, several subtask must be completed.
At first it is the graphical user interface and handling the camera which is managed through Android Activities and xml files defining the layout of the graphical interface.

Secondly, we deal with the calculation, which consist of image registration, keypoint detection, keypoint matching and solving the dense correspondence problem.
Image registration is implemented by using Sum of absolute differences described in \ref{sec:sad}.
Keypoints are detected with the Harris edge detector and described with ORB descriptor, which is an oriented variation of BRIEF (Binary Robust Independent Elementary Features) \cite{brief}.
%Extracted keypoints are with only
Each extracted keypoint we try to match with one extracted keypoint in the second image but only in the corresponding area determined due the previous image registration.
In this way we save calculation time since we do not have to compare every two detected keypoints.


\section{Project structure}
The project resources can be divided into four parts:
\begin{itemize}
\item{Activities, }
\item{Classes, }
\item{Structures, }
\item{xml files defining layout.}
\end{itemize}


